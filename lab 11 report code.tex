\documentclass[12pt]{article}
%\documentclass{scrartcl}
\usepackage{graphicx}
\usepackage{fancyhdr}
\rhead{Kate Antonov}
\lhead{11}
\pagestyle{fancy}
\usepackage{pythontex}
\usepackage{pythonhighlight}
\usepackage[T1]{fontenc}
\usepackage[scaled]{beramono}
\usepackage{listings}
\usepackage{capt-of}
\usepackage{grffile}
% Language and font encoding
\usepackage[english]{babel}
\usepackage[utf8x]{inputenc}
\usepackage[T1]{fontenc}
\usepackage{graphicx}
\usepackage{amsmath}
\usepackage{caption}
\usepackage{float}
\usepackage{caption}
\usepackage{subcaption}
\usepackage{rotating}
\usepackage{setspace}
\documentclass{article}
\captionsetup[figure]{font=small,skip=0pt}
\graphicspath{ {./images/} }
 

% Sets page size and margins
\usepackage[a4paper,top=3cm,bottom=2cm,left=3cm,right=3cm,marginparwidth=1.75cm]{geometry}

% Useful packages
\usepackage[colorinlistoftodos]{todonotes}
\usepackage[colorlinks=true, allcolors=blue]{hyperref}
\usepackage{listings}
\usepackage{gensymb}

%Line Spacing
\setstretch{1.5}

%-------------------Begin Editing Here---------------------
%Info for Title Page

\title{ECE 351-Section 51 \\ Z-Transform Operations
 \\ ------------------------------------------------------------------\\ 11 \\------------------------------------------------------------------}
 \author{Submitted by: \\  Kate Antonov}
 \date{November 19, 2019}
\begin{document}

%Make a Title Page
\vspace{\fill}

\maketitle

\vspace{\fill}
\thispagestyle{empty}
\clearpage

%Make Table of Contents
\clearpage
\thispagestyle{empty}
\tableofcontents
\clearpage

\section{Introduction}
This lab showed the user how to 
analyze a discrete system using Python and a function developed by Christopher Felton.
\newline
The equipment used included Spyder to program in Python. To test my code, click on this link: 
\url{https://github.com/kantonov477/ECE351_Code}
\newline
To see my lab report code, click on this link here: 
\url{https://github.com/kantonov477/ECE351_Reports}
\section{Equations}
\[y[k] = 2x[k] − 40x[k − 1] + 10y[k − 1] − 16y[k − 2]\]
\[16z^{-2}Y(z) + Y(z) + z^{-1}y[-1]+y[-2] - 10z^{-1}y(z) - y[-1] = 2X(z) - 40z^{-1}X(z) - x[-1]\]
\[Y(z)(16z^{-2} + 1 - 10Z^{-1}) = X(z)(2 - 40z^{-1})\]
\[H(z) = 2z(z - 20)/(z-8)(z-2)\]
\[H(z)/z = 2z(z - 20)/(z-8)(z-2) = A/(z-8) + B/(z-2)\]
\[A = -4\]
\[B = 6\]
\[H(z) = -4z/(z-8) + 6z/(z-2)\]
\[h[k] = -4(8^k)u[k] + 6(2^k)u[k]\]

\section{Methodology}
First, the transfer function was found by hand. h[k] was then found using fraction expansion. \newline
Scipy.signal.residuez() was then used to verify the partial fraction expansion done by hand.
The zplane() function was then used to obtain the pole-zero plot for H(z). Scipy.signal.freqz() was used to plot the magnitude and phase response of H(z).
\section{Results}
The z-plane plot:
\begin{figure}[H]
\includegraphics[scale=1]{Z pole plot.PNG}
  \caption{Z-Plane Plot}
  \end{figure}
The coded Magnitude and Phase Plot:
\begin{figure}[H]
\includegraphics[scale=0.85]{2nd plot.PNG}
  \caption{Coded Magnitude and Phase Plot}
  \end{figure}  
\section{Error Analysis}
It was noticed that the output for both magnitude and phase was not stable, which was to be expected.

\section{Questions}
1. Looking at the plot generated in Task 4, is H(z) stable? Explain why or why not.\newline
The transfer function is not stable because its range goes beyond the unit circle, also known as $2\pi$.
2. Leave any feedback on the clarity of lab tasks, expectations, and deliverables.\newline
I may suggest a more in-depth explanation of the stability of transfer functions in the z-domain? How does it compare to the reason in the s-domain?


\section{Conclusion}
In conclusion, this lab showed the user how to build and analyze discrete system in the z domain using both hand-derived equations and the library functions Python and Christopher Felton provides.

\end{document}